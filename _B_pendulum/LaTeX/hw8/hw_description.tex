For the inverted pendulum, do the following:
\begin{description}
\item[(a)] Using the principle of successive loop closure, draw a block diagram that uses PD control for both inner loop control and outer loop control. The input to the outer loop controller is the desired cart position $z^d$ and the output of the controller is the desired pendulum angle $\theta^d$.  The input to the inner loop controller is the desired pendulum angle $\theta^d$ and the output is the force $F$ on the cart.
\item[(b)] Focusing on the inner loop, find the PD gains $k_{P_\theta}$ and $k_{D_\theta}$ so that the rise time of the inner loop is $t_{r_\theta}=0.5$~seconds, and the damping ratio is $\zeta_{\theta}=0.707$.
\item[(c)] Find the DC gain $k_{DC_\theta}$ of the inner loop.
\item[(d)] Replacing the inner loop by its DC-gain and considering the outer loop, show that it is not possible to use the coefficient-matching approach to determine PD control gains. Further, convince yourself that using the DC-gain approximation for the inner loop, it is not possible to stabilize the position of the cart with PD control alone. We will investigate other methods to solve this problem in future assignments.
\item[(e)] Implement the inner-loop control design to control the pendulum angle in Simulink where the commanded pendulum angle is zero. Modify the Simulink diagram to include a saturation block on the force $F$ to limit the size of the input force on the cart to $F_{\max}=5$~N. Give the pendulum angle an initial condition of 10~deg and simulate the response of the system for 10~seconds to verify that the cart is able to balance the pendulum. 
\end{description}
